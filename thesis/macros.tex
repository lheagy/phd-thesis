% This file provides examples of some useful macros for typesetting
% dissertations.  None of the macros defined here are necessary beyond
% for the template documentation, so feel free to change, remove, and add
% your own definitions.
%
% We recommend that you define macros to separate the semantics
% of the things you write from how they are presented.  For example,
% you'll see definitions below for a macro \file{}: by using
% \file{} consistently in the text, we can change how filenames
% are typeset simply by changing the definition of \file{} in
% this file.
%
%% The following is a directive for TeXShop to indicate the main file
%%!TEX root = thesis.tex

\newcommand{\NA}{\textsc{n/a}}  % for "not applicable"
\newcommand{\eg}{e.g.,\ }   % proper form of examples (\eg a, b, c)
\newcommand{\ie}{i.e.,\ }   % proper form for that is (\ie a, b, c)
\newcommand{\etal}{\emph{et al}}

% Some useful macros for typesetting terms.
\newcommand{\file}[1]{\texttt{#1}}
\newcommand{\class}[1]{\texttt{#1}}
\newcommand{\latexpackage}[1]{\href{http://www.ctan.org/macros/latex/contrib/#1}{\texttt{#1}}}
\newcommand{\latexmiscpackage}[1]{\href{http://www.ctan.org/macros/latex/contrib/misc/#1.sty}{\texttt{#1}}}
\newcommand{\env}[1]{\texttt{#1}}
\newcommand{\BibTeX}{Bib\TeX}

% Define a command \doi{} to typeset a digital object identifier (DOI).
% Note: if the following definition raise an error, then you likely
% have an ancient version of url.sty.  Either find a more recent version
% (3.1 or later work fine) and simply copy it into this directory,  or
% comment out the following two lines and uncomment the third.
\DeclareUrlCommand\DOI{}
\newcommand{\doi}[1]{\href{http://dx.doi.org/#1}{\DOI{doi:#1}}}
%\newcommand{\doi}[1]{\href{http://dx.doi.org/#1}{doi:#1}}

% Useful macro to reference an online document with a hyperlink
% as well with the URL explicitly listed in a footnote
% #1: the URL
% #2: the anchoring text
\newcommand{\webref}[2]{\href{#1}{#2}\footnote{\url{#1}}}

% epigraph is a nice environment for typesetting quotations
\makeatletter
\newenvironment{epigraph}{%
    \begin{flushright}
    \begin{minipage}{\columnwidth-0.75in}
    \begin{flushright}
    \@ifundefined{singlespacing}{}{\singlespacing}%
    }{
    \end{flushright}
    \end{minipage}
    \end{flushright}}
\makeatother

% \FIXME{} is a useful macro for noting things needing to be changed.
% The following definition will also output a warning to the console
\newcommand{\FIXME}[1]{\typeout{**FIXME** #1}\textbf{[FIXME: #1]}}

% END


\newcommand{\SimPEG}{\textsc{SimPEG}\xspace}
\newcommand{\simpegEM}{\textsc{simpegEM}\xspace}

\newcommand{\Mesh}{\texttt{Mesh}\xspace}
\newcommand{\Survey}{\texttt{Survey}\xspace}
\newcommand{\DCSurvey}{\texttt{DCSurvey}\xspace}
\newcommand{\Problem}{\texttt{Problem}\xspace}
\newcommand{\DCProblem}{\texttt{DCProblem}\xspace}
\newcommand{\EMProblem}{\texttt{EMProblem}\xspace}
\newcommand{\Regularization}{\texttt{Regularization}\xspace}
\newcommand{\DataMisfit}{\texttt{DataMisfit}\xspace}
\newcommand{\Optimization}{\texttt{Optimization}\xspace}
\newcommand{\InvProblem}{\texttt{InvProblem}\xspace}
\newcommand{\Inversion}{\texttt{Inversion}\xspace}

\newcommand{\Mapping}{\texttt{Mapping}\xspace}
\newcommand{\ExpMap}{\texttt{ExpMap}\xspace}
\newcommand{\PropMap}{\texttt{PropMap}\xspace}
\newcommand{\Sources}{\texttt{Sources}\xspace}
\newcommand{\Receivers}{\texttt{Receivers}\xspace}
\newcommand{\Fields}{\texttt{Fields}\xspace}

\newcommand{\TDEM}{\texttt{TDEM}\xspace}
\newcommand{\FDEM}{\texttt{FDEM}\xspace}
\newcommand{\NSEM}{\texttt{NSEM}\xspace}
\newcommand{\BaseEM}{\texttt{BaseEM}\xspace}

\newcommand{\Exp}{\texttt{Exp}\xspace}
\newcommand{\InjectActiveCells}{\texttt{InjectActiveCells}\xspace}

\newcommand{\dobs}{\mathbf{d}_\text{obs}}
\newcommand{\m}{\mathbf{m}}
\newcommand{\du}{\mathbf{u}}
\newcommand{\mref}{\mathbf{m}_\text{ref}}
\newcommand{\dpred}{\mathbf{d}_\text{pred}}
\newcommand{\meshI}{\emph{mesh}$_I$\xspace}
\newcommand{\meshF}{\emph{mesh}$_F$\xspace}
\newcommand{\Wd}{\mathbf{W}_\text{d}}
\newcommand{\Wm}{\mathbf{W}_\text{m}}
\newcommand{\sm}{\mathbf{s}_m}
\newcommand{\se}{\mathbf{s}_e}

\newcommand{\deriv}[2]{\frac{\partial #1}{\partial #2}}

\newcommand{\minimize}[1]{\mathop{\hbox{minimize}}_{#1}}

\renewcommand{\div}{\nabla\cdot\,}
\newcommand{\grad}{\vec \nabla}
\newcommand{\curl}{{\vec \nabla}\times\,}
\newcommand {\J}{{\vec J}}
\renewcommand{\H}{{\vec H}}
\newcommand {\E}{{\vec E}}
\renewcommand {\S}{{\vec{S}}}
\newcommand{\dcurl}{{\mathbf C}}
\newcommand{\dgrad}{{\mathbf G}}
\newcommand{\Acf}{{\mathbf A_c^f}}
\newcommand{\Ace}{{\mathbf A_c^e}}
\newcommand{\St}{{\mathbf \Sigma_\tau}}
\newcommand{\T}{{\mathbf T}}
\newcommand{\Tt}{{\mathbf T_\tau}}
\newcommand{\diag}[1]{\,{\sf diag}\left( #1 \right)}

\newcommand{\M}{{\mathbf M}}
\newcommand{\MfMui}{{\M^f_{\boldsymbol{\mu^{-1}}}}}
\newcommand{\MfRho}{{\M^f_{\boldsymbol{\rho}}}}
\newcommand{\MeSigma}{{\M^e_{\boldsymbol{\sigma}}}}
\newcommand{\MeMu}{{\M^e_{\boldsymbol{\mu}}}}
\newcommand{\mui}{\mu^{-1}}

\newcommand{\MeSigInf}{{\M^e_{\sigma_\infty}}}
\newcommand{\MeSigO}{{\M^e_{\sigma_0}}}
\newcommand{\Me}{{\M^e}}
\newcommand{\Mes}[1]{{\M^e_{#1}}}
\newcommand{\Mee}{{\M^e_e}}
\newcommand{\Mej}{{\M^e_j}}
\newcommand{\BigO}[1]{\mathcal{O}\bigl(#1\bigr)}
\newcommand{\bE}{\mathbf{E}}
\newcommand{\bH}{\mathbf{H}}
\newcommand{\B}{\vec{B}}
\newcommand{\D}{\vec{D}}
\renewcommand{\H}{\vec{H}}
\newcommand{\s}{\vec{s}}
\newcommand{\bfJ}{\bf{J}}
\newcommand{\vecm}{\vec m}
\renewcommand{\Re}{\mathsf{Re}}
\renewcommand{\Im}{\mathsf{Im}}
\renewcommand {\j}  { {\vec j} }
\newcommand {\h}  { {\vec h} }
\renewcommand {\b}  { {\vec b} }
\newcommand {\e}  { {\vec e} }
\newcommand {\db}  { {\mathbf {b}} }
\newcommand {\de}  { {\mathbf {e}} }
\renewcommand {\d}  { {\vec d} }
\renewcommand {\u}  { {\mathbf u} }
\newcommand{\I}{\vec{I}}
