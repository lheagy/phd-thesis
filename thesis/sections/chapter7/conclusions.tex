\chapter{Conclusions and outlook}
\label{ch:conclusion}

Resource extraction and the storage of materials, such as CO$_2$, in the subsurface are current realities in our relationship with the earth. Making the best use of resources, such as the water required to perform hydraulic fracturing, and minimizing potential hazards, such as leakage of CO$_2$, requires that we have access to data which can ground and inform decisions. Electromagnetic geophysics has a role to play. In many cases, injected materials have different electrical conductivity than the host rock and the change in physical properties can be targeted with a non-invasive electrical or electromagnetic survey. A variety of EM surveys can be considered. Sources can be grounded or inductive, they can be positioned on the surface or within a borehole, and the data can consist of electric and/or magnetic field measurements. Each combination results in a different excitation and sensing of the earth. The main objective in designing a survey is to collect high quality data that are sensitive to the target. These data can then be inverted to obtain information about the subsurface structures of interest.

Inversions are an important tool for working with EM data, particularly when the goal is to extract information from subtle time-lapse signals. To formulate a meaningful inverse problem requires that: (a) we have data that are sensitive to the target of interest, (b) we have the ability to forward-simulate data and compute sensitivities, and (c) we can incorporate the necessary a-priori information and select a set of inversion parameters that allow the question-of-interest to be investigated. The success (or not) of the methods used to address each of these elements depends upon an understanding of the governing physics. In reservoir settings, the behavior of the currents, charges, fields and fluxes is significantly complicated by the presence of highly conductive, permeable steel-cased wells. As a result, much of my thesis focussed on developing a fundamental understanding of electrical and EM methods in the presence of steel-cased wells.

With respect to steel-cased wells, this thesis includes contributions in terms of numerical modelling as well as physical understanding of the behavior of electromagnetic fields and fluxes in settings with highly conductive, permeable wells. By discretizing Maxwell's equations on a 3D cylindrical mesh, I was able to design a mesh which finely discretizes the casing and captures 3D survey geometries without the number of cells in the mesh resulting in an intensive computation. I considered two forms of validation, both comparing values with published simulation results and comparing physical behaviors with that predicted in formative papers on DC and EM with steel cased wells. At DC, I demonstrated the transition between an approximately linear decay of currents away from the source along short wells to an exponential decay in longer wells. This behavior was predicted by both \cite{Schenkel1991, Kaufman1993} and has important implications in terms of testing approximations to the steel cased well. If the test is performed on a short well, there is minimal change in the behavior of the currents with changing conductivity. However, in longer wells, the product of the conductivity and the cross-sectional area controls the nature of the decay, and thus is the important quantity to conserve when approximating a steel cased well by a coarser conductivity structure. Related to the currents is the distribution of charges along the well. For short wells, the distribution is nearly constant while for longer wells, there is an exponential decay. There is also a significant charge increase near the end of the well. Although no closed-form solution exists for this behavior, \cite{Griffiths1997} similarly demonstrated this effect using an asymptotic analysis. When considering inverse problems in settings with steel-cased wells, this has the implication that the sensitivity is larger near the end of the well and as a result causes the inversion to shift structure closer to the end of the well.

The DC resistivity experiment is the starting point for building an understanding of the behavior of currents, charges, and electric fields when steel-cased wells are present. Using the application of casing integrity, I demonstrated that the location of the return electrode can be an important parameter in terms of survey design: the source and receiver lines can be oriented such that the coupling of the primary-field with the receivers is reduced as compared to a survey with a very distant return electrode. Reducing the coupling means that the secondary signals of interest comprise a larger percentage of the data, and therefore provide greater opportunity to extract meaningful information in an inversion. Our ability to detect a target of interest depends on many factors: the conductivity of the background, casing, and target, the location of the target, the position of the source electrodes, as well as the position and sensitivity of the receivers. The examples I demonstrated in Chapter \ref{ch:casing-dc} provide some indication of the relative importance of these parameters. Naturally, these details are site-specific and forward modelling is a critical tool for assessing the feasibility of a successful DC or EM survey in a given setting. The Jupyter Notebooks associated with this thesis can contribute to this task.

Electromagnetic problems introduce challenges both with respect to the time-variation of the fields and fluxes as well as the additional complexity caused by the significant magnetic permeability of the pipe. In a ``simple'' grounded-source time-domain EM experiment with one electrode connected to the top of the pipe and an offset return electrode, a shadow zone develops along the azimuth opposite to the source wire. This is caused by the image currents being channeled into the pipe. When permeability is included in the simulation, we saw that a poloidal current develops within the pipe. Along the inner wall, the current flows downwards, while along the outer wall, it flows upwards. The effects of permeability occur at later times and can alter radial electric field data measured at the surface by $>$ 100\% as compared to a pipe which is only conductive. This shows that in order to accurately simulate EM data collected in settings with steel-cased wells, it is critical that permeability be modelled. Depending on the survey and site parameters (e.g. base frequency, background conductivity, etc.), it may even influence data that are classically treated as DC; EM forward modelling can be used to assess if this is a concern.

To reduce computational load, particularly when considering the inverse problem where many forward simulations are required, it is of interest to represent the casing on a coarse mesh. At DC, we could achieve this by approximating the hollow-cased well by a solid cylinder that preserves the product of the conductivity and the cross sectional area of the pipe. However, in an EM experiment, the image current, which diffuses through time complicates the geometry of the currents and their interaction with the well. As a result, it appears that to design an appropriate approximation to the well, we may need to use a conductivity that varies through time and as a function of distance along the well. In the upscaling work I contributed to in \cite{Caudillo-Mata2017}, we showed that the values we estimated for a coarse-scale approximation to a fine-scale conductivity structure depended upon sensitivity. For example a different conductivity value might be obtained if we favored a value that replicated the behavior of the current density than if we favored a value that replicated the magnetic flux. Similarly, when approximating the casing in an EM experiment, the galvanic source currents provide a different excitation on the casing than the image current. Therefore, to capture each behavior on a coarse scale may require a more sophisticated approach than simply replacing the well with a constant conductivity. When permeability is included in the model, the behavior of the currents and magnetic flux are intertwined. It may be the case that a single scalar value can be used for the permeability, or it may be as complicated as conductivity and require that both a spatially and temporally varying model is necessary to accurately model the EM behavior. Designing a coarse-scale appropriate approximation to a steel-cased well is an avenue for future research. The development of the 3D cylindrical mesh simulation is an important contribution for this continuing work, as it provides a means for accurately discretizing the casing and performing simulations which include both permeability and conductivity.

The application of imaging hydraulic fractures was a primary motivator for the thesis. To create a sufficient contrast in electrical conductivity between the host rock and the fractures, I considered fracture operations conducted with an electrically conductive proppant and fluid. In order to estimate the electrical conductivity of a fractured volume of rock, I used a two-step homogenization process based on effective medium theory. In the first step, I estimate the conductivity of the proppant-fluid slurry and in the second, I estimate the conductivity of a volume of rock containing conductive cracks that are filled with the slurry. This approach provides a mathematical relationship between the concentration of conductive fractures and the coarse-scale conductivity. This relationship can be incorporated into an inversion, so that rather than inverting for log-conductivity, as is traditionally done in a DC or EM inversion, we can invert for a fracture concentration given a background conductivity model. Furthermore, it provides a mechanism for incorporating a-priori knowledge of the volume of injected proppant and fluid into the statement of the inverse problem. Although the workflow I adopted in this thesis uses effective medium theory to connect the concentration of proppant and fluid to the coarse-scale conductivity, the same methodology could be applied using an empirical relationship found in a lab-study.

The features of interest in an imaging problem for fractures are primarily geometric. For example, from an engineering standpoint, it is of interest to delineate the height and lateral extent of the propped volume of the reservoir. A smooth voxel inversion is then not necessarily an ideal approach as diffuse structures are recovered. Adopting a parametric representation of the fractured volume of rock is one trajectory for estimating its shape. I performed several parametric inversions for a DC resistivity experiment. In particular, the inversion results demonstrated some of the complications that arise due to the presence of the highly conductive steel casing. The parametric inversions in which log-conductivity was used to describe the properties of the target were highly sensitive to the starting model; the recovered conductivities varied by several orders of magnitude with small changes in the starting model parameters. If instead of using log-conductivity, we use effective-medium theory and invert for the fracture concentration in a parametric inversion, the results were more robust. Minor variations in the starting model did not drastically change the recovered model. The effective medium theory mapping is an approximately linear mapping between the fracture concentration and the conductivity, whereas the mapping from log-conductivity to conductivity is highly non-linear. This changes the sensitivities, and in a setting with such significant physical property contrasts, the non-linear log-conductivity mapping is not particularly stable. The combination of a parametric description of the fractured volume of rock and the effective-medium-theory mapping for the properties of the target resulted in reasonable model recoveries from the inversion. This was particularly true if the geometry of the starting model was somewhat representative of the true model.

Although parameterizing the model in terms of fracture concentration improved the stability of the inversion, there is still significant non-uniqueness. EM methods have significant potential to reduce some of this non-uniqueness because they provide richer information that comes with the variable excitation of the target through time. Exploring the use of EM inversions to image the propped-volume of a fracture is another avenue for future research. The software tools used to perform the DC inversions are implemented within the SimPEG framework. As such, the DC forward simulation component can readily be interchanged with an EM forward simulation and the same inversion machinery employed, which should reduce the overhead for future research along this trajectory.

The research conducted in this thesis required a significant amount of software to be developed. I needed to be able to forward-simulate Maxwell's equations at DC and in both the frequency and time-domains and view the results in terms of fields, fluxes and charges. For the EM simulations, permeability was included, and multiple formulations (E-B vs. H-J) were necessary for simulating inductive and grounded sources. For the inversion, I performed both voxel and parametric inversions in terms of log-conductivity and in terms of fracture-concentration through effective medium theory. This requires optimization machinery, computation of the sensitivities, and flexibility in how the inversion model is defined. Taken together, the composite of these tasks is an enormous effort and would be intractable if tackled in isolation. Further, much of the potential contribution to the broader community would be lost if the software had been written with the aim of ``getting the job done'' rather than written with the aim of re-use and adaptation by others. Working within the open-source ecosystem provides opportunities for researchers to leverage existing tools and to contribute new components into a larger framework where their value can be realized by a broader community. The work in this thesis has greatly benefited from work conducted by the SimPEG community and more broadly by the Jupyter and scientific Python communities. I in turn hope that the contributions I have made to the SimPEG ecosystem add long-term value to the geophysics community.

